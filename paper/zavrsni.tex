\documentclass[times, utf8, zavrsni]{fer}
\usepackage{booktabs}

\begin{document}

% TODO: Navedite broj rada.
\thesisnumber{000}

% TODO: Navedite naslov rada.
\title{Klasifikacija prometnih znakova}

% TODO: Navedite vaše ime i prezime.
\author{Matija Pavlović}

\maketitle

% Ispis stranice s napomenom o umetanju izvornika rada. Uklonite naredbu \izvornik ako želite izbaciti tu stranicu.
\izvornik

% Dodavanje zahvale ili prazne stranice. Ako ne želite dodati zahvalu, naredbu ostavite radi prazne stranice.
\zahvala{}

\tableofcontents

\chapter{Uvod}
Razvoj tehnologije u automobilskoj industriji u stopu prate i sve veći zahtjevi tržišta za novim sigurnosnim značajkama te značajkama koje doprinose udobnosti korištenja vozila. Novi modeli vozila tako postaju opremljeni značajnim brojem senzora na vanjskoj strani vozila i značajnim brojem ekrana i signalnih lampica u unutrašnjosti vozila. Kada sjednemo za upravljač novijih vozila sve češće možemo primijetiti da nas vozilo upozorava na prometne znakove, primjerice ograničenja brzine, zabrane pretjecanja, znakove obaveznog zaustavljanja itd. Razmotrimo li i činjenicu da ubrzano raste i broj vozila s određenim stupnjem autonomije pri vožnji postaje jasno da su sustavi koji u stvarnom vremenu detektiraju i klasificiraju prometne znakove postali izrazito važni u razvoju novih modela vozila. Cilj ovog završnog rada je demonstracija rada jednog takvog sustava uz detaljni opis primjene, problema s kojima se sustav može suočavati u stvarnim okolnostima, te opis implementacije sustava. U sklopu rada ću razviti model strojnog učenja temeljen na dubokoj konvolucijskoj mreži, obraditi skup podataka za treniranje i testiranje modela, te programski kod koji će koristiti kameru prijenosnog računala kako bi klasificirao prometne znakove.

\chapter{Pregled postojeće literature}

\chapter{Metodologija rada}
\section{Prikupljanje podataka za treniranje}
Skup podataka za treniranje se sastoji od

\chapter{Rezultati}

\chapter{Budući rad}

\chapter{Zaključak}
Zaključak.

\bibliography{literatura}
\bibliographystyle{fer}

\begin{sazetak}
Sažetak na hrvatskom jeziku.

\kljucnerijeci{Klasifikacija, računalni vid, strojno učenje, duboko učenje, duboke neuronske mreže, konvolucijske neuronske mreže, prometni znakovi, promet, DNN, CNN, CV, ML}

\end{sazetak}


\engtitle{Traffic sign classification}
\begin{abstract}
Abstract.

\keywords{Classification, Computer Vision, Machine Learning, Deep Learning, Deep Neural Networks, Convolutional Neural Networks, Traffic Signs, Traffic, DNN, CNN, CV, ML}

\end{abstract}

\end{document}
